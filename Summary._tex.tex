In 1915, Einstein formulated General Relativity as the law of gravity. Some of the most fascinating predictions of the theory include the existences of black holes and gravitational waves in our Universe. Remarkably, both of these predictions were directly observed by the LIGO observatories in 2015, exactly 100 years since the inception of the theory. Over the past several years, a network of gravitational-wave observatories have detected the coalescences of multiple binary black hole and binary neutron star systems. These detections have already transformed our understanding of astrophysics in many ways. Future detectors will achieve better sensitivities and observe over a larger range of frequencies, definitely unveiling much more about our Universe.


Another landmark achievement in physics over the last century is the development of the Standard Model of particle physics. This theory is extremely successful at predicting the outcomes of virtually all particle physics experiments, such as those conducted at the Large Hadron Collider in CERN. Despite its success, it is now an established fact that the Standard Model does not offer a complete description of Nature. For instance, various cosmological and astrophysical observations have shown that most matter in the Universe is a mysterious form of dark matter, which can only be described by physics beyond the Standard Model. In this thesis, I exploit gravitational wave observations to probe the dark side of our Universe. Specifically, I investigate the effect dark matter has on the gravitational waves emitted by merging black holes.


One of the most promising ways of probing dark matter with black holes is through a phenomenon called black hole superradiance. This is a process in which any ultralight dark matter bosons in our Universe would be spontaneously amplified around rotating black holes, thereby forming large boson clouds around these black holes. Because these boson clouds are large and carry significant amounts of energy, they can perturb their environments and leave important astrophysical signatures. In Chapter~\ref{sec:spectraAtom}, I presented the detailed analytic and numeric computations for various properties of these boson clouds. Since superradiance can occur for boson fields of any intrinsic spin, I described the computations for clouds that are made up of scalar and vector fields. These highly-precise results reveal all of the qualitative features of the clouds, which serve as crucial inputs for detailed studies of their observational implications. 


The boson clouds are easiest to detect when they are parts of binary systems. Specifically, their presence in binary systems can significantly perturb the binaries' trajectories, thereby affecting the associated gravitational-wave emissions. The detailed dynamics of the clouds and the backreaction on the orbits were investigated in Chapters~\ref{sec:Collider} and~\ref{sec:signatures}. Interestingly, it was found that the mathematics that describe these binaries' dynamical evolution are similar to those used to describe scattering processes in particle collider physics. I also explored the types of imprints the clouds could leave on the binaries' gravitational waveforms. Furthermore, I described how these imprints are sensitive to the masses and intrinsic spins of the underlying bosons --- direct observations of these waveform signatures would therefore not only imply the presence of the bosons in Nature, but also allow us to measure the microscopic properties of these new particles. This way of probing physics beyond the Standard Model makes binary black hole systems novel dark matter detectors. 


The boson clouds described above are by no means the only interesting probes of dark matter in binary systems. Other types of astrophysical objects can also arise in many physics beyond the Standard Model scenarios. Like the boson clouds, the dynamics of these new objects in binary systems could also be highly non-trivial ---  their associated gravitational waveforms are therefore also rich sources of information about the putative new physics at play. In Chapter~\ref{sec:search}, I presented an analysis that quantifies the extent to which we could detect these new binary signals with existing search methods, such as template bank searches with binary black hole waveforms. I concluded by describing how new search strategies and the construction of new template waveforms would be needed in order to detect these putative new signals in the observational data. This work motivates further advancements in current search strategies in order to fulfill our quest towards probing physics beyond the Standard Model using gravitational wave observations.

