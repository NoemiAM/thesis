Het doel van de natuurkunde is de onderliggende principes te onthullen die natuurlijke fenomenen beheersen. Deze wetten worden uitgedrukt via theoretische modellen, vaak in wiskundige termen (zoals de zwaartekrachtswet van Newton), en moeten worden getoetst aan experimenteel of observationeel bewijs (bijvoorbeeld door te meten hoe lang het duurt voordat een appel valt). Een cruciaal aspect van de natuurkunde is het kwantitatief vergelijken van deze theoretische modellen met empirische gegevens en metingen door middel van statistische analyse. 

Het primaire doel van dit proefschrift is nieuwe natuurkundige ontdekkingen mogelijk te maken door de statistische en computationele uitdagingen aan te pakken die zich voordoen in de vakgebieden van de astrofysica en kosmologie. Traditionele methoden in moderne astrofysische data-analyse zijn op steekproeven gebaseerde inferentietechnieken zoals Markov-chain Monte Carlo en geneste sampling-methoden. Deze strategieën vertrouwen echter vaak op benaderende likelihoods en hebben een belangrijk nadeel: de tijd die nodig is om tot convergentie te komen, schaalt slecht met de dimensionaliteit van de verkende parameterruimte. Om deze beperkingen te overwinnen, draagt dit proefschrift bij aan de ontwikkeling en vestiging van een alternatieve strategie op basis van nieuwe simulation-based inference (\gls*{sbi}) technieken, die de afgelopen jaren een significante ontwikkeling hebben doorgemaakt.

We begonnen in Hoofdstuk~\ref{cha:sbi} met het schetsen van het landschap van \gls*{sbi}-strategieën, waaronder traditionele methoden zoals approximate Bayesian computation. Vervolgens hebben we de verschillende neurale \gls*{sbi}-algoritmen beschreven, zoals neurale posterior schatting en neurale likelihood schatting, waarbij we ons met name hebben gericht op truncated marginal neural ratio estimation (\gls*{tmnre}), het voornaamste inferentie-algoritme dat in dit proefschrift wordt gebruikt. \Gls*{tmnre} is gebaseerd op drie belangrijke ingrediënten: neurale ratio schatting via classificatie, marginalisatie, en prior-truncatie. We benadrukten vervolgens de verschillende voordelen en valkuilen ten opzichte van de andere \gls*{sbi}-methoden, en bespraken teststrategieën. 

We hebben \gls*{tmnre} voor het eerst toegepast in Hoofdstuk~\ref{cha:lensing}, op de analyse van sterke-lensbeelden als probe voor donkere materie. Beginnend met een overzicht van sterke-lenswerking en het potentieel ervan om substructuren van donkere materie te onderzoeken, bespraken we de complexiteiten van de modellering die hierbij betrokken zijn, waaronder onzekerheden in de lens- en bronparameters, en hoe populaties van substructuren uit verschillende scenario’s voor donkere materie gemodelleerd kunnen worden. We hebben een \gls*{sbi}-pijplijn ontwikkeld om de cutoff-massa in de massafunctie van donkeremateriehalos rechtstreeks af te leiden uit gesimuleerde waarnemingen. Onze resultaten toonden aan dat \gls*{tmnre} nauwkeurige marginale en gerichte inferentie mogelijk maakt en daarmee traditionele computationele uitdagingen overwint. We lieten verder zien hoe hiërarchische inferentie kan worden toegepast om de cutoff-massa van donkere materie uit een dataset van lenzen te extraheren, wat de weg vrijmaakt voor toekomstige verbeteringen in de karakterisering van donkere materie met behulp van sterke-lenswaarnemingen. 

Vervolgens hebben we het \gls*{tmnre}-framework uitgebreid in Hoofdstuk~\ref{cha:anre}, waar we twee nieuwe technische bouwstenen voorstelden. Ten eerste ontwikkelden we autoregressieve ratio schatting met als doel om op robuuste wijze gecorreleerde hoog-dimensionale posteriors te schatten. Ten tweede stelden we een op slices gebaseerd genest sampling-algoritme voor om efficiënt zowel posterior samples als beperkte prior samples te genereren uit ratio-schatters, waarbij dit laatste instrumenteel is voor sequentiële inferentie. 

In Hoofdstuk~\ref{cha:cosmo} hebben we het inverse probleem aangepakt dat zich richt op de overgang van niet-lineaire, niet-lokale beschrijvingen van dichtheidsvelden in het late universum naar Gaussische kosmologische beginvoorwaarden. Hiervoor hebben we autoregressieve Gaussische likelihood-schatting gebruikt om de voorwaardelijke afhankelijkheden tussen pixels in het dichtheidsveld te modelleren. Posterior sampling wordt uitgevoerd via een Gibbs-sampling-algoritme op basis van exacte data-augmentatie, wat zorgt voor efficiënte verkenning van hoog-dimensionale parameterruimten. De voorgestelde aanpak combineert computationele efficiëntie met toepasbaarheid op generieke, niet-differentieerbare forward simulators, waardoor deze geschikt is voor bredere astrofysische en kosmologische data-analysetaken. 

Tot slot hebben we in Hoofdstuk~\ref{cha:detection} het probleem van puntbron-detectie en populatieparameterinferentie in hemelkaarten behandeld. We ontwikkelden een goed interpreteerbaar \gls*{sbi}-framework (omdat het lijkt op componenten van traditionele survey-analyseworkflows) dat, voor zover de auteurs weten, voor het eerst consistente puntbron-detectie en populatieparameterinferentie mogelijk maakt, zowel voor gedetecteerde als sub-drempelbronnen. Dit werd mogelijk gemaakt door brondetectie te definiëren als een nieuwe en hoog-dimensionale vorm van prior-truncatie om gedetecteerde bronnen op te nemen in het simulatiemodel. 

Over het geheel genomen was het doel van dit proefschrift om het potentieel van \gls*{sbi} voor astrofysische data-analyse te benadrukken, en dit framework te positioneren als een essentieel onderdeel van de moderne toolkit voor natuurkundige data-analyse.  

\textit{Vertaald door Dion Noordhuis.} 