The increasing quantity and complexity of astrophysical datasets pose significant challenges for conventional analysis methods. In recent years, there has been a remarkable development of simulation-based inference (SBI) algorithms, now applied across a wide range of astrophysical problems. These techniques offer a qualitative shift in our approach to statistical inference and a promising avenue to fully exploit the data potential.
In this thesis, we aim to highlight the key benefits of SBI methods for astrophysical data analysis, and argue for their central role in the modern physics data analysis toolkit. First, we provide an overview of traditional and neural network-based SBI implementations, focusing particularly on the specific algorithm that will be mostly employed in this thesis, truncated marginal neural ratio estimation. We then illustrate its effectiveness with three concrete examples: the analysis of strong gravitational lenses as a probe of dark matter substructures, the reconstruction of cosmological initial conditions from late time density fields, and the analysis of point-sources in sky maps. 
